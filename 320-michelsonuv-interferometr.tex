\documentclass[a4paper]{article}

\usepackage[czech]{babel} %https://github.com/michal-h21/biblatex-iso690
\usepackage[
   backend=biber      % if we want unicode 
  ,style=iso-numeric % or iso-numeric for numeric citation method          
  ,babel=other        % to support multiple languages in bibliography
  ,sortlocale=cs_CZ   % locale of main language, it is for sorting
  ,bibencoding=UTF8   % this is necessary only if bibliography file is in different encoding than main document
]{biblatex}

\usepackage[utf8]{inputenc}
\usepackage{fancyhdr}
\usepackage{amsmath}
\usepackage{amssymb}
\usepackage[left=2cm,right=2cm,top=2.5cm,bottom=2.5cm]{geometry}
\usepackage{graphicx}
\usepackage{pdfpages}
\usepackage{url}

\usepackage{siunitx}
\sisetup{locale = DE}  %, separate-uncertainty = true    kdybych chtel +/-

\usepackage{float}
\newfloat{graph}{htbp}{grp}
\floatname{graph}{Graf}
\newfloat{tabulka}{htbp}{tbl}
\floatname{tabulka}{Tabulka}

\renewcommand{\thefootnote}{\roman{footnote}}

\pagestyle{fancy}
\lhead{Praktikum III - (20) Stavba Michelsonova interferometru a ověření jeho funkce}
\rhead{Vladislav Wohlrath}
\author{Vladislav Wohlrath}

\bibliography{source}

\begin{document}

\begin{titlepage}
\includepdf[pages={1}]{./graficos/titlelist.pdf}
\end{titlepage}

\section*{Pracovní úkoly}
\begin{enumerate}
\item Změřte divergenci laserového svazku. Průměry svazku změřte na milimetrovém papíru i měřičem profilu svazků a obě metody porovnejte.
\item Sestavte Galileův teleskop. Změřte, kolikrát rozšiřuje průměr svazku, a výsledek porovnejte s výpočtem rozšíření ze známých ohniskových délek čoček.
\item Sestavte Michelsonův interferometr. Vysvětlete princip vzniku interferenčních proužků.
\item Pozorujte, popište a vysvětlete změny v interferenčním obrazci při:
        \begin{enumerate}
        \item naklánění zrcadla Z4,
        \item posunu zrcadla Z3 mikrometrickým šroubem,
        \item vkládání skla do svazku ve čtyřech polohách kolem děliče svazku
        \item ohřátí vzduchu v různých místech průchodu svazku.
        \end{enumerate}

\end{enumerate}

%Teoretická část
\section*{Teoretická část}
Laserový svazek má přibližně gaussovský profil a rozbíhá se.
Rozbíhavost svazku charakterizujeme divergencí\cite{skripta}
\begin{equation} \label{e:div}
d=\frac{D_2-D_1}{s} \,,
\end{equation}
kde $D_1$ je průměr svazku u výstupního otvoru a $D_2$ je průměr svazku ve vzdálenosti $s$.
Minimální dosažitelná divergence $d_m$ lze odhadnout\cite{skripta}
\begin{equation} \label{e:dmin}
d_m=\frac{2\lambda}{D_1} \,,
\end{equation}
kde $\lambda=\SI{632.8}{\nm}$ je vlnová délka použitého světla.


K úpravě šířky svazku použijeme Galileův teleskop. Teleskop je tvořen jednou rozptylnou a jednou spojnou čočkou. V našem případě je ohnisková vzdálenost rozptylné čočky $f_1=\SI{-25}{\mm}$ a spojné čočky $f_2=\SI{200}{\mm}$.
Příčné zvětšení (rozšíření svazku) je pak dáno jejich poměrem 
\begin{equation} \label{e:rozsireni}
\beta=\num{8} \,.
\end{equation}



%Výsledky měření
\section*{Výsledky měření}
Průměr svazku v různých vzdálenostech měřený oběma metodami je uveden v tabulce \ref{t:div}.
Jako $D_1$ jsme vzali v obou případech hodnotu pro $s=\SI{5}{\cm}$.
Pro výpočet divergence jsme se rozhodli použít jako $D_2$ pouze hodnotu s největší vzdáleností $s$, protože ostatní hodnoty jsou zatíženy vysokou chybou. Při měření měřičem profilu svazku uvádíme hodnoty \emph{beam width clip}.

\begin{tabulka}[htbp]
\centering
\begin{tabular}{cc|cc}
\multicolumn{2}{c|}{ručně} & \multicolumn{2}{c}{měřič profilu} \\
$s$ (\si{\cm}) & $D$ (\si{\mm}) & $s$ (\si{\cm}) & $D$ (\si{\mm}) \\ \hline
5 & \num{1.0(2)} & 5 & \num{0.65(3)} \\
18 & \num{1.0(2)} & 28 & \num{0.70(2)} \\
66 & \num{1.5(2)} & 64 & \num{0.99(3)} \\
155 & \num{2.5(4)} & 154 & \num{2.07(10)} \\
251 & \num{4.0(5)} & 250 & \num{3.39(10)} \\ \hline
\multicolumn{2}{c|}{$d=\num{0.0012(4)}$} & \multicolumn{2}{c}{$d=\num{0.0011(1)}$} \\
\multicolumn{2}{c|}{$d_{min}=\num{0.0013(3)}$} & \multicolumn{2}{c}{$d_{min}=\num{0.0019(1)}$} \\
\end{tabular}
\caption{Měření průměru svazku}
\label{t:div}
\end{tabulka}


Změřili jsme průměr svazku po průchodu Galileovým teleskopem (jen ručně na milimetrovém papíru)
\begin{equation*}
D_G=\SI{9.0(5)}{\mm} \,,
\end{equation*}
to znamená, že rozšíření svazku je \num{9(2)}. To se shoduje s výpočtem ze známých ohniskových délek \eqref{e:rozsireni}.
Svazek byl rovnoběžný, v celé pozorované oblasti jsme nepozorovali měřitelnou změnu průměru svazku.



\subsection*{Michelsonův interferometr}
Sestavili jsme Michelsonův interferometr. Laserový svazek jsme děličem svazku rozdělili na dva a oba jsme zrcadlem odrazili zpět do děliče svazku. Výsledné svazky spolu v prostoru za děličem interferovali a my jsme je promítli na stínítko.


Interferující svazky nebudou přesně rovnoběžné, ale budou mírně sbíhavé. Při interferenci sbíhavých svazků, které svírají úhel $\alpha$, vzniknou podle \cite{skripta} na stínítku interferenční plošky s rozestupem
\begin{equation}
\Delta x=\frac{\lambda}{2 \sin(\alpha/2)} \,.
\end{equation}
Experiment tomu kvalitativně odpovídal.


Při naklánění zrcadla Z4 měníme úhel mezi svazky a jejich společnou rovinu, takže se mění rozestup proužků a jejich směr.


Při posunu zrcadla Z3 měníme optickou dráhu jednoho ze svazků.
To podle \cite{skripta} odpovídá posunu proužků ve směru na ně kolmém.


Při vložení skla před dělič svazku nebo před stínítko změníme optickou dráhu obou svazků totožně a na interferenčním obrazci se to nijak neprojeví. Pokud sklo vložíme mezi dělič svazku a jedno ze zrcadel Z3 nebo Z4, změníme optickou dráhu příslušného svazku, zatímco optická dráha druhého interferujícího svazku zůstane nezměněna. V tom případě nastane případ podobný posouvání zrcadla Z3, při vložení zrcadla se proužky posunou. Následným nakláněním zrcadla můžeme dodatečně měnít optickou dráhu svazku (a posouvat proužky). Pokud je sklo nějakým způsobem nehomogenní nebo je jeho povrch nerovný, proužky se mírně zakřiví. Také dochází k rozptylu a interferenční proužky jsou nepatrně rozostřené.


Při zahřívání vzduchu v místě průchodu svazku mezi děličem svazku a jedním ze zrcadel Z3 nebo Z4 dochází ke zkřivení interferenčních proužků a jejich kolísání v důsledku časově i prostorově proměnlivého indexu lomu vzduchu v ohřívaném místě.

%Diskuze výsledků
\section*{Diskuze}
Měření průměru svazku na milimetrovém papíru je neobyčejně nepřesné. Z hodnot průměru svazku je vidět, že jsme při měření na milimetrovém papíru považovali svazek za širší než při měření detektorem. Divergence vyšla pro obě metody shodně a menší než minimální dosažitelná divergence. Poměr změřené divergence a minimální dosažitelné divergence je citlivý na přesnou definici průměru svazku, nestačí jen vždy konzistentně uvažovat stejnou část svazku.
Z tohoto důvodu nepovažujeme výpočet divergence za směroplatný, jelikož ta část paprsku, jejíž průměr jsme měřili, byla určena spíše náhodně a ne v souladu s teorií.

Skutečné zvětšení Galileova teleskopu se shoduje s teoretickou hodnotou v rámci chyby měření, která je však velmi vysoká.

Stavba Michelsonova interferometru proběhla v pořádku.

Všěchny změny v interferenčním obrazci při provádění pracovního úkolu 4 byly ve shodě s teoretickými předpovědmi.

%Závěr
\section*{Závěr}
Změřili jsme divergenci laserového svazku na milimetrovém papíře a měřičem profilu svazků (viz tabulka \ref{t:div}).

Sestavili jsme Galileův teleskop a změřilli jeho příčné zvětšení \num{9(2)}.

Sestavili jsme Michelsonův interferometr. Vysvětlili jsme princip vzniku interferenčních proužků a změny v interferenčním poli při různých změnách podmínek (viz Výsledky).


\printbibliography[title={Seznam použité literatury}]

\end{document}