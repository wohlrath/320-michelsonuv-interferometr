\section*{Diskuze}
Měření průměru svazku na milimetrovém papíru je neobyčejně nepřesné. Z hodnot průměru svazku je vidět, že jsme při měření na milimetrovém papíru považovali svazek za širší než při měření detektorem. Divergence vyšla pro obě metody shodně a menší než minimální dosažitelná divergence. Poměr změřené divergence a minimální dosažitelné divergence je citlivý na přesnou definici průměru svazku, nestačí jen vždy konzistentně uvažovat stejnou část svazku.
Z tohoto důvodu nepovažujeme výpočet divergence za směroplatný, jelikož ta část paprsku, jejíž průměr jsme měřili, byla určena spíše náhodně a ne v souladu s teorií.

Skutečné zvětšení Galileova teleskopu se shoduje s teoretickou hodnotou v rámci chyby měření, která je však velmi vysoká.

Stavba Michelsonova interferometru proběhla v pořádku.

Všěchny změny v interferenčním obrazci při provádění pracovního úkolu 4 byly ve shodě s teoretickými předpovědmi.