\section*{Teoretická část}
Laserový svazek má přibližně gaussovský profil a rozbíhá se.
Rozbíhavost svazku charakterizujeme divergencí\cite{skripta}
\begin{equation} \label{e:div}
d=\frac{D_2-D_1}{s} \,,
\end{equation}
kde $D_1$ je průměr svazku u výstupního otvoru a $D_2$ je průměr svazku ve vzdálenosti $s$.
Minimální dosažitelná divergence $d_m$ lze odhadnout\cite{skripta}
\begin{equation} \label{e:dmin}
d_m=\frac{2\lambda}{D_1} \,,
\end{equation}
kde $\lambda=\SI{632.8}{\nm}$ je vlnová délka použitého světla.


K úpravě šířky svazku použijeme Galileův teleskop. Teleskop je tvořen jednou rozptylnou a jednou spojnou čočkou. V našem případě je ohnisková vzdálenost rozptylné čočky $f_1=\SI{-25}{\mm}$ a spojné čočky $f_2=\SI{200}{\mm}$.
Příčné zvětšení (rozšíření svazku) je pak dáno jejich poměrem 
\begin{equation} \label{e:rozsireni}
\beta=\num{8} \,.
\end{equation}

